\documentclass[a4paper]{article}
\setlength{\oddsidemargin}{-2cm} \setlength{\topmargin}{-20mm}
\setlength{\textwidth}{20cm} \setlength{\textheight}{30cm}
\pagestyle{empty}

\def\DJ{\leavevmode\setbox0=\hbox{D}\kern0pt
 \rlap{\kern.04em\raise.188\ht0\hbox{-}}D}
\def\dj{\leavevmode\setbox0=\hbox{d}\kern0pt
 \rlap{\kern.215em\raise.46\ht0\hbox{-}}d}
\begin{document}

\begin{center}
\textbf{Konstrukcija kompilatora - Septembar 2018.}\\
\end{center}

\begin{enumerate}

\item Unaprediti ve\' c postoje\' ci interpreter koji podr\v zava cele
  neozna\v cene brojeve u kompilator sa mogu\' cno\v s\' cu prevo\dj
  enja na LLVM.

\begin{enumerate}

\item (ve\' c ura\dj eno) Interpreter ima mogu\' cnost \v stampanja i ra\v cunanja
  vrednosti celobrojnih izraza. Nad brojevima su podr\v zani bitovski
  operatori i ($and$), ili ($or$), ekskluzivno ili ($xor$), negacija
  ($not$) i levo i desno siftovanje ($shl$ i $shr$), sa prioritetom i
  asocijativno\v s\' cu kao u programskom jeziku C. Brojevi se zadaju
  u dekadnom ili heksadekadnom sistemu.

\begin{verbatim}
main() {
  # stampanje dekadno
  print 4 or 1226 and 0xf;
  print not 23 and 0xff;
  print 16 shl 4 shr 1;
  print 16 shl (4 shr 1)
}
\end{verbatim}

\item (ve\' c ura\dj eno) Interpreter ima mogu\' cnost memorisanja vrednosti izraza u
  promenljivu \v cije se ime sastoji samo od malih slova.
\begin{verbatim}
main() {
  # dodela
  set tmp to 127 shl 2;
  print tmp or 1024 shl 1
}
\end{verbatim}

\item (ve\' c ura\dj eno) Rezultat izraza je mogu\' ce ispisati i u heksadekasnom sistemu,
  u zavisnosi od promenljive $flag$. Ako promenljiva nije definisana,
  ili ako joj je vrednost 0, ispis je dekadni, ina\v ce je
  heksadekadni.

\begin{verbatim}
main() {
  # stampanje heksadekadno
  set flag to not 0;
  print tmp or 1024 shl 1
}
\end{verbatim}

\item Dodati mogu\' cnost da program sadr\v zi jo\v s neku funkciju,
  osim $main()$, pri \v cemu se pretpostavlja da sve one vra\' caju
  celobrojnu vrednost (vrednost poslednjeg izraza), a broj argumenta
  mo\v ze biti $0$ ili vi\v se.
  
\begin{verbatim}
f(x, y, z) {                                      main() {
  set a to x or y;                                  set tmp to call f(8 or 2, 4, 1);
  a or z                                            print tmp;
}                                                   set tmp to call one();
one() {                                             print tmp
  1                                               }
}
\end{verbatim}

\item Dodati petlje.

\begin{verbatim}
rotate_left(x, n) {                               main() {
  set first_bit_mask to 1 shl 31;                   set flag to not 0;
  set i to 0;                                       set tmp to 10;
  while (i < n) {                                   print tmp;
    set first_bit to x and first_bit_mask;          set tmp to call rotate_left(tmp, 4);
    set x to x shl 1 or first_bit shr 31;           print tmp
    set i to i + 1                                }
  };
  x
}
\end{verbatim}

\end{enumerate}

\end{enumerate}

\end{document}
