\documentclass[a4paper]{article}
\setlength{\oddsidemargin}{-2cm} \setlength{\topmargin}{-3cm}
\setlength{\textwidth}{20cm} \setlength{\textheight}{28cm}
\pagestyle{empty}

\def\DJ{\leavevmode\setbox0=\hbox{D}\kern0pt
 \rlap{\kern.04em\raise.188\ht0\hbox{-}}D}
\def\dj{\leavevmode\setbox0=\hbox{d}\kern0pt
 \rlap{\kern.215em\raise.46\ht0\hbox{-}}d}
\begin{document}

\begin{center}
\textbf{Konstrukcija kompilatora - Jul 2017.}\\
prakti\v cni deo
\end{center}

\begin{enumerate}
\item U prilo\v zenom direktorijumu nalazi se kompilator jezika za
  opisivanje geometrijskih objekata u ravni, sa mogu\' cno\v s\' cu
  prevo\dj enja na LLVM IR (\verb|--llvm|), objektni kod
  (\verb|--compile|), ali i interpretacije (bez argumenata komandne
  linije). Jezik je strogo tipiziran, tako da zahteva deklaraciju svih
  promenljivih, pre njihove upotrebe. \verb"vector" predstavlja tip
  podatka kojim se predstavljaju dvodimenzionali vektori u
  ravni. Deklarisanje ovog tipa je mogu\' ce na slede\' ci na\v cin:
\begin{verbatim}
vector v = [1,3.14], v1 = [-2, 1], v2, v3;
\end{verbatim}
Jezik podr\v zava \v stampanje vektora i promenu vrednosti ve\' c
deklarisanim promenljivama. Dozvoljene operacije nad vektorima su
sabiranje, oduzimanje i mno\v zenje konstantom.
\begin{verbatim}
print v;                                        [1, 3.14]
v2 = [0,1];
print 3*v+v2;                                   [3, 10.42]
v3 = v2 - [10,10];
vector c = -2*[-0.5,-0.5];
\end{verbatim}
Unaprediti jezik dodavanjem jo\v s jednog tipa podataka:
\verb"transformation" predstavlja transformacije vektora u
ravni. Tipovi transformacija su homotetija sa zadatim koeficijentom,
rotacija i kompozicija vi\v se transformacija. Omogu\' citi
deklaraciju transformacija i njihovu primenu na vektore.  Imena
promenljivih koje predstavljaju transformacije po\v cinju znakom \$,
pa zatim malim slovom, koje eventualno prate neke cifre.


U nastavku sledi predlo\v zen na\v cin realizacije zadatka. Prvo
implementirati interpretaciju.
\begin{enumerate}
\item Homotetija
\begin{verbatim}
vector v = [1,2]; 
print v;
print scaling(3)(v);
print scaling(6)([0, 1] + [0, 1]);
\end{verbatim}
\item Rotacija
\begin{verbatim}
vector v = [1,2]; 
print v;
print rotation(3.14)(v);
print rotation(6.28)([0, 1] + [0, 1]);
\end{verbatim}
\item Deklaracija
\begin{verbatim}
vector v = [1,2]; 
print v;
transformation $f = rotation(3.14);
transformation $g = scaling(3);
print $f([0, 1]);
print $f(v);
print $g([0, 1]);
print $g(v);
\end{verbatim}
\item Kompozicija
\begin{verbatim}
vector c = -2*[-0.5,-0.5];
transformation $f = rotation(3.14);
transformation $g = scaling(2);
transformation $y = scaling(-1) * $g * $f;
print $y*scaling(3)(c);
\end{verbatim}
\end{enumerate}

A zatim i kompilaciju istim redosledom, vode\' ci se istim test
primerima u delovima (e), (f), (g) i (h).

\end{enumerate}

\end{document}
